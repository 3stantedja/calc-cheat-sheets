\documentclass[10pt]{extarticle}
\usepackage{problems,lg_jsylvest}
\usepackage{multicol}

\geometry{
	margin=0.7in,
	headheight=1in
}

% define some additional operators
\DeclareMathOperator{\arccot}{arccot}
\DeclareMathOperator{\arcsec}{arcsec}
\DeclareMathOperator{\arccsc}{arccsc}

\theoremstyle{definition}
\newtheorem*{theorem}{Theorem}

\title{Cheat Sheet for Midterm 1 (MATH 144)}
\date{8 February 2024}

\begin{document}
	\section{Algebra \& trigonometry}
	\subsection*{Special angles}
		% tables are hell—send help.
		\begin{tabular}{r c c c c c c c c c}
			\toprule
			\textbf{DEG} & \(\ang{0}\) & \(\ang{30}\) & \(\ang{45}\) & \(\ang{60}\) & \(\ang{90}\) & \(\ang{120}\) & \(\ang{135}\) & \(\ang{150}\) & \(\ang{180}\) \\
			\textbf{RAD} & \(0\) & \(\frac{\pi}{6}\) & \(\frac{\pi}{4}\) & \(\frac{\pi}{3}\) & \(\frac{\pi}{2}\) & \(\frac{2\pi}{3}\) & \(\frac{3\pi}{4}\) & \(\frac{5\pi}{6}\) & \(\pi\) \\ \midrule
			\(\sin\theta\) & \(0\) & \(\frac{1}{2}\) & \(\frac{\sqrt{2}}{2}\) & \(\frac{\sqrt{3}}{2}\) & \(1\) & \(\frac{\sqrt{3}}{2}\) & \(\frac{\sqrt{2}}{2}\) & \(\frac{1}{2}\) & \(0\) \\
			\(\cos\theta\) & \(1\) & \(\frac{\sqrt{3}}{2}\) & \(\frac{\sqrt{2}}{2}\) & \(\frac{1}{2}\) & \(0\) & \(-\frac{1}{2}\) & \(-\frac{\sqrt{2}}{2}\) & \(-\frac{\sqrt{3}}{2}\) & \(-1\) \\
			\(\tan\theta\) & \(0\) & \(\frac{\sqrt{3}}{3}\) & \(1\) & \(\sqrt{3}\) & DNE & \(-\sqrt{3}\) & \(-1\) & \(-\frac{\sqrt{3}}{3}\) & \(0\) \\
			\bottomrule
		\end{tabular}

		For the resulting values of trig functions to be positive, it follows a quadrant with the order \emph{all, sine, cosine, tangent} —
		starting at the top-right and moving anticlockwise.

	\begin{multicols}{2}
		\subsection*{Cotangent \& some reciprocal identities}
		\begin{align*}
			\csc\theta &= \frac{1}{\sin\theta} & \cot\theta &= \frac{1}{\tan\theta} \\
			\sec\theta &= \frac{1}{\cos\theta} & &= \frac{\cos\theta}{\sin\theta}
		\end{align*}

		\subsection*{Derivative of the inverse trig functions}
		\begin{align*}
			\frac{d}{d\theta} \arcsin{(\theta)} &= \frac{1}{\sqrt{1 - \theta^2}} & \frac{d}{d\theta} \arccos{(\theta)} &= -\frac{1}{\sqrt{1 - \theta^2}} \\
			\frac{d}{d\theta} \arctan{(\theta)} &= \frac{1}{1 + \theta^2} & \frac{d}{d\theta} \arccot{(\theta)} &= -\frac{1}{1 + \theta^2} \\
			\frac{d}{d\theta} \arcsec{(\theta)} &= \frac{1}{\theta \sqrt{\theta^2 - 1}} & \frac{d}{d\theta} \arccsc{(\theta)} &= -\frac{1}{\theta \sqrt{\theta^2 - 1}}
		\end{align*}

		\subsection*{Pythagorean identities}
		\begin{align*}
		\sin^2{\theta} + \cos^2{\theta} &= 1 \\
		\tan^2{\theta} + 1 &= \sec^2{\theta} \\
		1 + \cot^2{\theta} &= \csc^2{\theta}
		\end{align*}

		\subsection*{Sum and difference formulae}
		\begin{align*}
			\sin{(\theta \pm \phi)} &= \sin{a}\cos{b} \pm \cos{a}\sin{b} \\
			\cos{(\theta \pm \phi)} &= \cos{a}\cos{b} \mp \sin{a}\sin{b} \\
			\tan{(\theta \pm \phi)} &= \frac{\tan{\theta} \pm \tan{\phi}}{1 \mp \tan{\theta}\tan{\phi}}
		\end{align*}

		\subsection*{Double-angle \& half-angle formulae}
		These build upon the sum-difference formulae.
		\begin{align*}
			\sin{2\theta} &= 2\sin{\theta}\cos{\theta} \\
			\cos{2\theta} &= \cos^2{\theta} - \sin^2{\theta} \\
			\tan{2\theta} &= \frac{2\tan{\theta}}{1 - \tan^2\theta} \\
			\sin{\frac{\theta}{2}} &= \pm \sqrt{\frac{1 - \cos\theta}{2}} \\
			\cos{\frac{\theta}{2}} &= \pm \sqrt{\frac{1 + \cos\theta}{2}}\\
			\tan{\frac{\theta}{2}} &= \pm \sqrt{\frac{1 - \cos\theta}{1 + \cos\theta}}
		\end{align*}

		\subsection*{Logarithms \& some of their properties}
		\[y = \log_{b}x \lgcequiv x = b^{y}\]
		\begin{align*}
			\log_{b}{b} &= 1 & \log_{b}{b^x} &= x \\
			b^{\log_{b}{x}} &= x & \log_{b}{1} &= 0
		\end{align*}
		\begin{align*}
			\log_{b}{xy} &= \log_{b}{x} + \log_{b}{y} & \log_{b}{a^x} &= x \log_{b}{a} \\
			\log_{b}{\frac{x}{y}} &= \log_{b}{x} - \log_{b}{y} & \log_{b}{\sqrt[x]{a}} &= \frac{\log_{b}{a}}{x}
		\end{align*}

		\subsection*{Inequalities, absolute values, \& some of their properties}
		% segments copied straight from Paul's Online Math Notes
		If \(a < b\) then \(a + c\) and \(a - c < b - c\)

		If \(a < b\) and \(c > 0\) then \(ac < bc\) and \(\frac{a}{c} < \frac{b}{c}\)

		If \(a < b\) and \(c < 0\) (i.e. \(c\) is negative) then \(ac > bc\) and \(\frac{a}{c} > \frac{b}{c}\)

		If \(b\) is positive, then:
		\begin{align*}
			\abs{p} &= b &\lgcimplies && p = -b \; \text{or} \; p = b \\
			\abs{p} &< b &\lgcimplies && -b < p < b \\
			\abs{p} &> b &\lgcimplies && p < -b \; \text{or} \; p > b
		\end{align*}
	\end{multicols}
	\pagebreak

	\section{Limits}
	\begin{multicols}{2} \raggedcolumns \setcounter{unbalance}{10}
		\subsection*{Basics}
		\begin{definition}[rough]
			\(\lim_{x \to a}{f(x)} = L\) means that as we approach \(x\) closer to \(a\) \emph{but not equal to} \(a\), 
			the value for \(f(x)\) can get arbitrarily closer to \(L\).
		\end{definition}

		Note that \(\lim_{x \to a}{f(x)} = L\) \textbf{iff} \(\lim_{x \to a^{-}}{f(x)} = L\) (LR limit) 
		and \(\lim_{x \to a^{+}}{f(x)} = L\) (RH limit) are both true.

		\subsection*{Properties}
		Assuming that the limit exists, then all the following is true:
		\begin{align*}
			\lim_{x \to a} k = k && \lim_{x \to a} x = a
		\end{align*}
		\begin{equation*}
			\lim_{x \to a} f(x) = f(a) \; \text{if the function \(f\) is \emph{continuous} at \(x = a\)}
		\end{equation*}
		\begin{align*}
			\lim_{x \to a}(f(x) \pm g(x)) &= \lim_{x \to a}f(x) \pm \lim_{x \to a}g(x) \\
			\lim_{x \to a}(k \cdot f(x)) &= k \lim_{x \to a}f(x) \\
			\lim_{x \to a}(f(x) \cdot g(x)) &= \lim_{x \to a}{f(x)} \cdot \lim_{x \to a}{g(x)} \\
			\lim_{x \to a} \frac{f(x)}{g(x)} &= \frac{\lim_{x \to a} f(x)}{\lim_{x \to a} g(x)} \;\;\;\; (g(x) \neq 0) \\
			\lim_{x \to a}f(x)^n &= \left( \lim_{x \to a}{f(x)}\right)^n \\
		\end{align*}

		If \(\displaystyle \lim_{x \to a} f(x) = L\), \(\lim_{x \to L} g(x) = K\) s.t. \(g(L) = K\), then
		\begin{equation*}
			\lim_{x \to a} g(f(x)) = K.
		\end{equation*}

		\subsection*{Squeeze theorem}
		\begin{theorem}
		If \(f(x) \leq g(x) \leq h(x)\) for all \(x\) near \(x = a\) (except possibly at \(x = a\)), and
		\[\lim_{x \to a}{f(x)} = \lim_{x \to a}{h(x)} = L,\] then \[\lim_{x \to a}{g(x)} = L\]
		\end{theorem}

		\subsection*{Some important-to-remember limits}
		\begin{align*}  % this one is also pretty hacky; i'd prefer using something like array for next time.
			\lim_{x \to 0} \frac{1}{x} &= \text{DNE (1/0)} & \lim_{x \to 0} \frac{1}{x^2} &= \infty \\
			\lim_{x \to 0} \frac{\sin{x}}{x} &= 1 & \lim_{x \to 0} \frac{\cos{x} - 1}{x} &= 0 \\
			\lim_{x \to 0} (1 + x)^{1/x} &= e & \lim_{x \to 0} \frac{e^x - 1}{x} &= 1 \\
			\lim_{x \to 0} \sin{\frac{1}{x}} &= \text{DNE (chaos)}
		\end{align*}
	\end{multicols}
	\pagebreak

	\section{Acknowledgements}
	The cheat sheet is partly inspired by (and copied from) Paul's Online Math Notes (\url{https://tutorial.math.lamar.edu/}); in particular his various cheat sheets.


\end{document}
