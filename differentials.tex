\documentclass[main.tex]{subfiles}

\begin{document}
\section{Differential equations}
\ifSubfilesClassLoaded{%
\begin{multicols}{2} \raggedcolumns \setcounter{unbalance}{10}
}{}
	
	\subsection*{Separable ODEs}
	\begin{definition}[basic]
	A first-order differential equation is \emph{separable} if it can be written as \[\frac{dy}{dx} = g(x)f(y).\]
	\end{definition}
	
	\subsubsection*{Solving separable ODEs}
	\begin{enumerate}
	\item Find the constant solution where \(f(y) = 0\) (or equivalents thereof).
	\item Assuming that \(f(y) \neq 0\), rewrite the equation to the form of \[\frac{dy}{f(y)} = g(x) \ dx.\]
	\item Integrate both sides. If it's integrable analytically, then the resulting equation is the general solution.
	\item If possible, solve for \(y(x)\).
	\end{enumerate}

	\subsection*{Linear ODEs}
	\begin{definition}[basic]
	A differential equation is \emph{linear} if it can be written as \[\frac{dy}{dx} + P(x)y = Q(x).\]
	\end{definition}

	\subsubsection*{Solving linear ODEs}
	\begin{enumerate}
	\item Find the integrating factor \[I(x) = e^{\int P(x) \ dx}\]
	\item Multiply both sides of the equation such that \[\frac{d(I(x) y)}{dx} = I(x)Q(x)\]
	\item Integrate both sides and solve for \(y.\) This yields a general solution.
	\item Use the initial conditions to get a specific solution for the given problem.
	\end{enumerate}

	\subsection*{Phase line analysis}
	Mainly for \emph{autonomous} differential equations (i.e. forms similar to \(\frac{dy}{dx} = f(y)\).)
	The idea is to infer the qualitative behaviour of the solutions \(y(x)\) by inspecting the differential equation, identifying:
	\begin{enumerate}
	\item points where \(f(y) = 0\), meaning solution is constant,
	\item regions where either \(f(y) > 0\) or \(f(y) < 0\) --- meaning solution is either increasing or decreasing, respectively.
	\end{enumerate}

	If values near a constant are tending towards each other, the constant is said to be a stable equilibrium solution. Otherwise, it's unstable.

	% [TODO]: add some example forms of separable differential equations?
	\subsection*{Some forms of rate equation problems}

	\subsection*{Growth models}
	\subsubsection*{Exponential growth model}
	Given \[\frac{dP}{dt} = \lambda P\] where \(P(0) = P_0\) and \(\lambda > 0\), the solution to the initial-value problem is
	\[P(t) = P_0e^{\lambda t}.\]

	\subsubsection*{Logistic growth model}
	Given a carrying capacity of \(M\), a variation of the exponential growth model can be derived as \[\frac{dP}{dt} = \lambda P \left(1 - \frac{P}{M}\right),\] where \(P(0) = P_0.\) This requires phase-line analysis.

	\subsubsection*{Logistic growth model with linear harvesting.}
	Assuming that we perform constant harvesting of \(H\), a variation of the previous model can be derived as \[\frac{dP}{dt} = \lambda P \left(1 - \frac{P}{M}\right) - H,\] where \(P(0) = P_0.\)


\ifSubfilesClassLoaded{%
\end{multicols}
}{}

\end{document}