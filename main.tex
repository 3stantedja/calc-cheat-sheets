\documentclass[9pt]{extarticle}
\usepackage{problems,lg_jsylvest}
\usepackage{multicol}
%\usepackage{titlesec}
\usepackage{subfiles}

\newgeometry{
	margin=0.6in,
	headheight=1in
}

% remove header and footers so we have more space.
\pagestyle{plain}

% define some additional operators
\DeclareMathOperator{\arccot}{arccot}
\DeclareMathOperator{\arcsec}{arcsec}
\DeclareMathOperator{\arccsc}{arccsc}
%\DeclareMathOperator{\tanh}{tanh}
%\DeclareMathOperator{\coth}{coth}
\DeclareMathOperator{\sech}{sech}
\DeclareMathOperator{\csch}{csch}
\DeclareMathOperator{\arcsinh}{arcsinh}
\DeclareMathOperator{\arccosh}{arccosh}
\DeclareMathOperator{\arctanh}{arctanh}
\DeclareMathOperator{\arccoth}{arccoth}
\DeclareMathOperator{\arcsech}{arcsech}
\DeclareMathOperator{\arccsch}{arccsch}

\theoremstyle{definition}
\newtheorem*{theorem}{Theorem}

\theoremstyle{remark}
\newtheorem*{note}{Note}

\title{Cheat Sheet for MATH 144}
\date{18 April 2024}

\begin{document}
	\subfile{algebra_trig}
	\pagebreak
	\subfile{limits}
	\pagebreak
	\subfile{differentiation}
	\pagebreak
	\subfile{integration}

	\section{Acknowledgements}
	The cheat sheet is partly inspired by (and copied from) Paul's Online Math Notes (\url{https://tutorial.math.lamar.edu/}); in particular his various cheat sheets.

	Additionally, the completing the square section uses material from Wikipedia (\url{https://en.wikipedia.org/wiki/Completing_the_square}.)

	This cheat sheet also uses material from the APEX Calculus textbook (\url{https://www.apexcalculus.com/}).


\end{document}
