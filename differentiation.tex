\documentclass[main.tex]{subfiles}

\begin{document}
\section{Differentiation}
\begin{multicols}{2} \raggedcolumns \setcounter{unbalance}{10}
	\subsection*{Basics}
	\begin{definition}[informal]
	A function \(f\) is differentiable at \(x = a\) if \[f'(a) = \lim_{h \to 0} \frac{f(a+h) - f(a)}{h}\] exists.
	Additionally, \(f\) is differentiable on an open interval \((a, b)\) if \(f\) is differentiable at every point in the interval.
	\end{definition}
	\begin{note}
	If the function \(f\) is differentiable at \(x = a\), then it will be continuous at \(x = a\).
	\emph{However}, a function \(f\) that is continuous at \(x = a\) \textbf{may or may not} be differentiable at \(x = a\).
	\end{note}

	\subsection*{Common derivatives}
	\subsubsection*{Inverse functions}
	\begin{equation*}
		(f^{-1})' (x) = \frac{1}{f'(f^{-1}(x))}
	\end{equation*}
	\subsubsection*{Trig functions}
	\begin{align*}
		\frac{d}{d\theta} \sin{(\theta)} &= \cos{(\theta)} & \frac{d}{d\theta} \cos{(\theta)} &= -\sin{(\theta)} \\
		\frac{d}{d\theta} \tan{(\theta)} &= \sec^2{(\theta)} &  \frac{d}{d\theta} \cot{(\theta)} &= -\csc^2{(\theta)} \\
		\frac{d}{d\theta} \sec{(\theta)} &= \sec{(\theta)}\tan{(\theta)} & \frac{d}{d\theta} \csc{(\theta)} &= -\csc{(\theta)}\cot{(\theta)} \\
		\frac{d}{d\theta} \arcsin{(\theta)} &= \frac{1}{\sqrt{1 - \theta^2}} & \frac{d}{d\theta} \arccos{(\theta)} &= -\frac{1}{\sqrt{1 - \theta^2}} \\
		\frac{d}{d\theta} \arctan{(\theta)} &= \frac{1}{1 + \theta^2} & \frac{d}{d\theta} \arccot{(\theta)} &= -\frac{1}{1 + \theta^2} \\
		\frac{d}{d\theta} \arcsec{(\theta)} &= \frac{1}{\theta \sqrt{\theta^2 - 1}} & \frac{d}{d\theta} \arccsc{(\theta)} &= -\frac{1}{\theta \sqrt{\theta^2 - 1}}
	\end{align*}

	\subsubsection*{Logarithms}
	\begin{align*}
		\frac{d}{dx} a^x &= a^x \ln{(a)} & \frac{d}{dx} e^x &= e^x \\
		\frac{d}{dx} \ln{(x)} &= \frac{1}{x}, \ \ x > 0 & \frac{d}{dx} \ln{|x|} &= \frac{1}{x}, \ \ x \neq 0
	\end{align*}

	\subsection*{Critical points \& concavity}
	\subsubsection*{Critical points}
	For a function \(f(x)\), \(x = a\) is a critical point if either
	\begin{enumerate}
		\item \(f'(a) = 0\), or
		\item \(f'(a)\) DNE
	\end{enumerate}

	\subsubsection*{Concavity \& inflection points}
	For a function \(f(x)\) with an interval \(I\),
	\begin{enumerate}
		\item if \(f''(x) > 0 \ \forall x \in I\), then \(f(x)\) is concave up on \(I\),
		\item if \(f''(x) < 0 \ \forall x \in I\), then \(f(x)\) is concave down on \(I\), and
		\item if \(f''(a) = 0\), then \(x = a\) is an inflection point, where its concavity changes.
	\end{enumerate}

	\subsection*{Fermat's theorem}
	\begin{theorem}
	If \(f(x)\) has a relative (or local) extrema at \(x = a\), then \(x = a\) is a critical point of \(f(x)\)
	\end{theorem}

	\subsection*{Extreme value theorem}
	\begin{theorem}
	If \(f(x)\) is continuous on the closed interval \([a, b]\), then there exists numbers \(c\) and \(d\) s.t.
	\begin{enumerate}
		\item \(a \leq c\), \(d \leq b\),
		\item \(f(c)\) is the absolute maximum in \([a, b]\),
		\item \(f(d)\) is the absolute minimum in \([a, b]\).
	\end{enumerate}
	\end{theorem}

	\subsection*{Mean value theorem}
	\begin{theorem}
	Suppose that \(f(x)\) is continuous on the closed interval \([a, b]\) and is differentiable on the open interval \((a, b)\).
	Then, there exists a number \(c \in (a, b)\) s.t. \[f'(c) = \frac{f(b) - f(a)}{b - a}.\]
	\end{theorem}

	\subsection*{Linear approximation}
	For a linear approximation \[f(x) \approx f(a) + f'(a)(x - a),\] we can apply a linearization of \[f(x) \approx f(a) + f'(a)(x - a).\]

	\subsection*{Newton's method}
	\begin{definition}
	If \(x_n\) is the nth guess for the solution of \(f(x) = 0\), then the \((n + 1)\)th guess would be \[x_{n+1} = x_n - \frac{f(x_n)}{f'(x_n)},\] provided that \(f'(x_n)\) exists.
	\end{definition}

	\subsection*{Solving problems with derivatives}
	Always start by reading the problem carefully; drawing a diagram if necessary or helpful. Then, follow the instructions for each types of problems.
	\subsubsection*{Related rates}
	\begin{enumerate}
		\item Introduce notation. Assign symbols to all quantities that are functions of time.
		\item Express those notations and rate of changes in terms of derivatives.
		\item Write an equation, then differentiating both sides w.r.t. \(t\).
		\item Substitute given information to new equation, and solve.
	\end{enumerate}

	\subsubsection*{Optimization}
	\begin{enumerate}
		\item Introduce notations. Assign a symbol to the quantity to be optimized, alongside other relevant symbols to represent the unknowns.
		\item Write an equation for the quantity in terms of other unknowns. If the quantity is expressed in more than one variable, substitute and rearrange in terms of one of the variables.
		\item Find the absolute maximum/minimum of the equation; in a closed interval do the closed interval method, otherwise find the local max/min and justify why it might be the absolute max/min of the equation.
	\end{enumerate}


\end{multicols}
\end{document}